% 3 Generalized Linear Model

\par The generalized linear model in this study makes a few advancements that
draws from earlier statistical analyses. Firstly, the diagnosis analysis finds
that outliers in the BOLD data are nonfrequent and not particularly out-of-hand
so the process used here skips the step of dropping outlier volumes and/or
trials. A subsequent decision made here is to substitute the raw data in favor
of the smoothed filtered data produced in the earlier smoothing analysis, as the
spatial correlation within the data is more pronounced in the latter.

\par \indent This generalized linear model approach requires producing a new
design matrix containing as many rows as volumes in the BOLD data and exactly
six columns. The first column from the left contains all ones. The second,
third, and fourth columns contain the convolved neural time courses for
parametric gain, parametric loss, and the distance from indifference,
respectively, that were computed in the convolution analysis. The fifth column
contains 240 equally-spaced and sequential values between -1 and 1, inclusive,
that represents a \textit{linear drift} component, included to partially or
fully cancel systematic linear offset of the data over time. The last column
contains a \textit{quadratic drift} component, computed as the absolute
deviation among squared values of linear drift, included for reasoning similar
to that for the \textit{linear drift} component. Explicitly, the \textit{linear
drift} component is defined to be
\[
\mathrm{ linear \ drift \ component } \equiv \mathbf{ L } = \left[ \; -1 \quad
-\frac{ 237 }{ 239 } \quad \ldots \quad \frac{ 237 }{ 239 } \quad 1 \; \right] ,
\]
and the \textit{quadratic drift} component to be
\[
\mathrm{ quadratic \ drift \ component } = \mathbf{ L }^{ 2 } -
\bar{ \mathbf{ L }^{ 2 } } .
\]

\par \indent Next, a threshold must be determined to identify which voxels in
the data are located inside the brain. The procedure used here sets the
threshold at 80th percentile with respect to voxel means. This means that the
mean signal strength of each voxel is computed over the dimension of time. Those
voxels whose mean signal strength falls in the top twenty percent of voxels'
mean signal strengths are determined to represent a location inside the brain.
In order to save time and memory, computations are done only for voxels located
inside the brain.

\par \indent After the fitting of a generalized linear model, each voxel that is
determined to be inside the brain has three values associated with it for each
of the six regressors (including the intercept): the regression coefficient, and
the t-statistic and corresponding p-value that indicate the regression
coefficient's statistical significance.

\par \indent Lastly, the neural loss aversion is computed for each voxel located
in the brain as the sum of the additive inverse of the regression coefficient
for parametric loss and the additive inverse of the regression coefficient for
parametric gain, or
\[
\hat{ \lambda }_{ neural } = - \hat{ \beta }_{ loss } - \hat{ \beta }_{ gain }.
\]
This study was particularly interested in the voxel associated with the B
ventral striatum, which had MNI coordinates of $\left( 3.6 , 6.3 , 3.9 \right)$
in millimeters. This particular value of neural loss aversion was saved and set
aside for later.
