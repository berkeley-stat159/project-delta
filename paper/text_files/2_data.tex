% TeX file: 2 Data
\par The original study recruited sixteen right-handed, English-speaking
individuals between the ages of 19 and 28. There is confusion about the sexes of
the individuals as the supplement to the published article reports seven males
and nine females, while the data made publicly available reports eight males and
eight females. It is known that all subjects were physically healthy and free of
neurological and psychiatric history, and that their informed consent was
acquired prior to their participation in the study.
\par \indent These subjects participated in three runs of 85-86 trials of a
single task. This task involved offering the subject a wager with an equal
chance of winning or losing known amounts of money. Potential gains were in even
denominations between \$10 and \$40, while potential losses were in integer
denominations between \$5 and \$20. The participants were prompted to either
strongly accept, weakly accept, weakly decline, or strongly decline the wager in
each trial. At the same time that investigators collected these behavioral
responses, data pertaining to neural activity was recorded via fMRI. This data
set is available for download from OpenFMRI.org under the name ``Mixed-gambles
task'' (accession number ds005), and offers both the neural activity data as
well as the behavioral data.
\par \indent The neural activity data for each run is saved under the filename
\texttt{bold.nii.gz}, each file containing 240 volumes obtained over the course
of eight minutes. Each volume further contains exactly 139264 voxels, being 64
points in length, 64 points in width, and 34 points in depth. As per usual with
blood-oxygen-level dependent (BOLD) analysis, each voxel also corresponds to a
signal at a given time. A larger signal indicates increased oxygenated blood
flow to the corresponding are of the brain, which this paper will refer to as
``activation.''
\par \indent The behavior data for each run is stored separately under the alias
\texttt{behavdata.txt}. Each file contains a table for the run, in which each
trial of the task is stored as a row. Each row then contains seven elements,
referred to in the file as \texttt{onset}, \texttt{gain}, \texttt{loss},
\texttt{PTval}, \texttt{respnum}, \texttt{respcat}, and \texttt{RT}. The first
three elements respectively denote the time at which the trial was presented,
the proposed monetary reward given in the event the subject wins the trial, and
the proposed monetary loss suffered in the event the subject lost the trial. The
fourth element is a standardized value indicating the relative expected gain
from the wager. The fifth element represents the subject's response: \texttt{1}
for strong acceptance, \texttt{2} for weak acceptance, \texttt{3} for weak
rejection, \texttt{4} for strong rejection, and \texttt{0} for nonresponse. The
sixth element contains the patient response converted to a binary variable:
\texttt{0} for rejection, \texttt{1} for acceptance, and \texttt{-1} in the
outstanding case of nonresponse. The final element is the time it took the
subject to submit a response, measured in seconds and with precision to within a
millisecond.