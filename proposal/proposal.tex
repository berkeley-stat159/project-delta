\documentclass[11pt]{article}
\bibliographystyle{siam}

\title{The neural basis of loss aversion in decision making under risk}
\author{
  Kong, Victor\\
  \texttt{VictorKong94}
  \and
  Li, Ce\\
  \texttt{karenceli}
  \and
  Liu, Anna\\
  \texttt{liuanna}
  \and
  Qin, Weidong\\
  \texttt{j170382276}
  \and
  Xia, Yunfei\\
  \texttt{yfxia}
}

\begin{document}
\maketitle

By exploring the published fMRI paper \textit{The neural basis of loss aversion in decision making under risk} and the accompanying data ($http://openfmri.s3.amazonaws.com/tarballs/ds005_raw.tgz$), our team decides to investigate the neural response when individuals decided whether to accept or reject gambles that offered a 50/50 chance to win or lose. By the prospect theory by Daniel Kahneman, the paper examines the loss aversion by analyzing how individual's brains react differently when they are facing potential gains or loss. Humans are more sensitive to loss than gains. The goal of the study is whether loss aversion reflects the engagement of distinct emotional processes when potential losses are considered. The paper investigates whether individual differences in brain activity during decision-making are related to individual differences in behavior. Researchers used fMRI images to analyze the whole brain and to identify regions where the neural response to gains or losses was correlated with loss aversion. The paper illustrates how neuroimaging is used to directly test the prediction from prospect theory that risk aversion for mixed gambles can be attributed to enhanced sensitivity to losses. After downloading the data file, we had our initial check in the BOLD data of first three subjects and plot their 2D brain images. We first chose the first volumn and slicied them at the same index to compare their similarities and differences. 

We plan to reproduce the process in analyzing imaging data by first identifying which brain regions correlated with the size of potential gain or loss using parametric regressors. In order to validate the conjuction analysis between increasing activity for gains and decresing activity for losses in a set of regions, we plan to create by averaging paramter estimates versus baseline within each cluster in the conjunction map for each of the 16 cells in the gain/loss matrix. We next investigate whether individual differences in brain activity during decision-making were related to individual differences in behavior, using whole-brain analyses to identify regions where neural reponse to gains or losses was correlated with behavioral loss aversion. A scatterplot of correspondence between neural loss aversion and behavioral 
loss aversion in the most activited brain region will be helpful. A set of statistical analysis such as linear regression and hypothesis testing for parameters in the regression model will be conducted to help analyzing the data. 

\bibliography{proposal}

\end{document}
