\documentclass[11pt]{article}
\bibliographystyle{siam}

\title{The title of your project proposal}
\author{
  Kong, Victor\\
  \texttt{VictorKong94}
  \and
  Lee, Karen\\
  \texttt{karenceli}
  \and
  Liu, Anna\\
  \texttt{liuanna}
  \and
  Qin, Eric\\
  \texttt{j170382276}
  \and
  Xia, Sophia\\
  \texttt{yfxia}
}

\begin{document}
\maketitle

The Neural Basis of Loss Aversion in Decision-Making Under Risk
\cite{tom2007neural} investigated the neurobiology of decision-making in humans.
Certain areas of the brain are hypothesized to be sensitive to gain and others
sensitive to loss. This study involved a single task in which subjects indicated
how likely (strongly accept, weakly accept, weakly reject, strongly reject) they
were to partake in a 50/50 chance gamble given certain monetary stakes, while
being monitored via fMRI. For incentive, participants were offered a stipend
beforehand and told that one decision from each of three scanning runs would be
applied to their total payout.

The uncompressed archive contains 2.03 GB of data, which comes from a total of
sixteen subjects, half of which are male and the other half female, all between
the ages of 19 and 28. The data for each subject is also further divided into
four subarchives: \texttt{BOLD} contains data that has already been processed
and cleaned as well as a previous analyst's complete results, \texttt{anatomy}
contains multiple collections of high-resolution fMRI data; \texttt{behav} the
parameters of each trial as well as the response given by the subject; and
\texttt{model} normalized data for trials' stakes and patient responses.

We propose to begin our investigation by corroborating Tom et al's study. Our
approach will involve first establishing baseline or mean levels of neural
activity over time for each spatial position and each subject studied in the
brain fMRIs. Then, we intend to compare subjects' neural activity levels to
these baseline values to determine the relative correlations between a different
trials' stakes and sites of increased or decreased activity.

Other ideas we would like to explore include how males and females differ in
their main sites of neural activity when prompted to make a decision (in
particular, we are interested in whether males and females weigh stakes
differently). We might also be interested in exploring whether or not subjects'
age correlated with their decision-making process (if older participants saw
stronger responses in certain areas as their brains, in particular their frontal
lobes, have had more time to develop).

\bibliography{proposal}

\end{document}
